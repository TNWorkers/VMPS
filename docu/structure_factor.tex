\documentclass[12pt,paper=a4,BCOR=16mm]{article}
\usepackage{ifxetex}
\ifxetex
	\usepackage{fontspec}
	\usepackage{libertine}
\else
	\usepackage[T1]{fontenc}
	\usepackage[utf8]{inputenc}
	\usepackage{lmodern}
\fi

\usepackage{datetime}
\usepackage{amsmath}
\usepackage{amsthm}
\usepackage{amssymb}
\usepackage{subfigure}
\usepackage{graphicx}
\usepackage{caption}
\usepackage[dvips,letterpaper,left=1.0in,top=0.35in,right=1.0in,bottom=0.75in]{geometry}
\usepackage{dsfont}
\usepackage{tikz}
\usepackage{algorithm}
%\usepackage{url}
\usepackage[style=authoryear,
            maxbibnames=99, %list all authors in fullcite
            %citestyle=authoryearbrackets,
            uniquename=false, %no first names
            uniquelist=false,  %no first names
            %sorting=nymt, %sorting according to name, year, month, t? (custom scheme in authoryearbrackets.cbx)
            firstinits=true,
            url=false,doi=false,isbn=false
           ]{biblatex}
\AtEveryCite{\renewcommand{\mkbibnamelast}[1]{\mkbibemph{#1}‌​}} %last names in italics
\renewbibmacro{in:}{} %suppresses "in:" before journal
\bibliography{/afs/physnet.uni-hamburg.de/users/th1_po/rrausch/PHDTEX/main/Rausch_2016.bib}

% enumerate bibliography items despite 'authoryear' citation style
\defbibenvironment{bibliography}
  {\enumerate
     {}
     {\setlength{\leftmargin}{\bibhang}%
      \setlength{\itemindent}{-\leftmargin}%
      \setlength{\itemsep}{\bibitemsep}%
      \setlength{\parsep}{\bibparsep}}}
  {\endenumerate}
  {\item}

%\bibliographystyle{eng-own}
\usepackage[noend]{algpseudocode}
\usetikzlibrary{decorations.markings}
\usetikzlibrary{decorations.pathmorphing}
\usetikzlibrary{decorations.pathreplacing}

%\usepackage{hyperref}
%\hypersetup{colorlinks=true,bookmarksopen=true,bookmarksopenlevel=3,allcolors=blue}
%\usepackage{hypcap}
\usepackage{pdfpages}

\setlength{\parindent}{0ex}
\setlength{\parskip}{1.3ex plus 0.2ex minus 0.2ex}

%\newcommand{\dirIntro}{/afs/physnet.uni-hamburg.de/users/th1_po/rrausch/PHDTEX/Intro/}
%\newcommand{\dirED}{/afs/physnet.uni-hamburg.de/users/th1_po/rrausch/PHDTEX/ED/}
%\newcommand{\dirLanczos}{/afs/physnet.uni-hamburg.de/users/th1_po/rrausch/PHDTEX/Lanczos/}
%\newcommand{\dirDMRG}{/afs/physnet.uni-hamburg.de/users/th1_po/rrausch/PHDTEX/DMRG/}
%\newcommand{\dirBetheAnsatz}{/afs/physnet.uni-hamburg.de/users/th1_po/rrausch/PHDTEX/BetheAnsatz/}
%\newcommand{\dirBetheAES}{/afs/physnet.uni-hamburg.de/users/th1_po/rrausch/PHDTEX/BetheAES/}
%\newcommand{\dirAESeq}{/afs/physnet.uni-hamburg.de/users/th1_po/rrausch/PHDTEX/AESeq/}
%\newcommand{\dirDoublonDyn}{/afs/physnet.uni-hamburg.de/users/th1_po/rrausch/PHDTEX/DoublonDyn/}
%\newcommand{\dirTwoParticleSpec}{/afs/physnet.uni-hamburg.de/users/th1_po/rrausch/PHDTEX/TwoParticleSpec/}

% hyperlinks for citations with biblatex:

\DeclareCiteCommand{\cite}
  {\usebibmacro{prenote}}
  {\usebibmacro{citeindex}%
   \printtext[bibhyperref]{\usebibmacro{cite}}}
  {\multicitedelim}
  {\usebibmacro{postnote}}

\DeclareCiteCommand*{\cite}
  {\usebibmacro{prenote}}
  {\usebibmacro{citeindex}%
   \printtext[bibhyperref]{\usebibmacro{citeyear}}}
  {\multicitedelim}
  {\usebibmacro{postnote}}

\DeclareCiteCommand{\parencite}[\mkbibparens]
  {\usebibmacro{prenote}}
  {\usebibmacro{citeindex}%
    \printtext[bibhyperref]{\usebibmacro{cite}}}
  {\multicitedelim}
  {\usebibmacro{postnote}}

\DeclareCiteCommand*{\parencite}[\mkbibparens]
  {\usebibmacro{prenote}}
  {\usebibmacro{citeindex}%
    \printtext[bibhyperref]{\usebibmacro{citeyear}}}
  {\multicitedelim}
  {\usebibmacro{postnote}}

\DeclareCiteCommand{\footcite}[\mkbibfootnote]
  {\usebibmacro{prenote}}
  {\usebibmacro{citeindex}%
  \printtext[bibhyperref]{ \usebibmacro{cite}}}
  {\multicitedelim}
  {\usebibmacro{postnote}}

\DeclareCiteCommand{\footcitetext}[\mkbibfootnotetext]
  {\usebibmacro{prenote}}
  {\usebibmacro{citeindex}%
   \printtext[bibhyperref]{\usebibmacro{cite}}}
  {\multicitedelim}
  {\usebibmacro{postnote}}

\DeclareCiteCommand{\textcite}
  {\boolfalse{cbx:parens}}
  {\usebibmacro{citeindex}%
   \printtext[bibhyperref]{\usebibmacro{textcite}}}
  {\ifbool{cbx:parens}
     {\bibcloseparen\global\boolfalse{cbx:parens}}
     {}%
   \multicitedelim}
  {\usebibmacro{textcite:postnote}}

\definecolor{asparagus}{rgb}{0.53, 0.66, 0.42}
\definecolor{goldenrod}{rgb}{0.85, 0.64, 0.125}
\newcommand{\qcite}[1]{\colorbox{asparagus}{[#1]}} %quick citation
\newcommand{\note}[1]{\colorbox{goldenrod}{[#1]}} %quick note
\newcommand{\qnote}[1]{\colorbox{goldenrod}{[#1]}} %quick note

\newcommand{\bi}{\begin{itemize}}
\newcommand{\ei}{\end{itemize}}
\newcommand{\be}{\begin{equation}}
\newcommand{\ee}{\end{equation}}
\newcommand{\bs}{\begin{split}}
\newcommand{\es}{\end{split}}

\let\vaccent=\v % rename builtin command \v{} to \vaccent{}
\renewcommand{\v}[1]{\ensuremath{\mathbf{#1}}} % for vectors
\newcommand{\ket}[1]{\big| #1 \big>} % for Dirac bras
\newcommand{\bra}[1]{\big< #1 \big|} % for Dirac kets
\newcommand{\braket}[2]{\big<#1 \vphantom{#2} \big| #2 \vphantom{#1} \big>} % for Dirac brackets
\newcommand{\matrixel}[3]{\big< #1 \vphantom{#2#3} \big| #2 \big| #3 \vphantom{#1#2} \big>} % for Dirac matrix elements
\newcommand{\pd}[2]{\frac{\partial #1}{\partial #2}} % for partial derivatives
\renewcommand{\d}[2]{\frac{d #1}{d #2}} % for derivatives
\newcommand{\gv}[1]{\ensuremath{\mbox{\boldmath$ #1 $}}} % for vectors of Greek letters
\renewcommand{\div}[1]{\gv{\nabla} \cdot #1} % for divergence
\newcommand{\curl}[1]{\gv{\nabla} \times #1} % for curl
\renewcommand{\=}[1]{\stackrel{#1}{=}} % for putting numbers above =
\newcommand{\avg}[1]{\big< #1 \big>} % for average

\newcommand{\lr}[1]{\left(#1\right)}
\newcommand{\lrsq}[1]{[#1]}
\newcommand{\sqrtfrac}[2]{\sqrt{\frac{#1}{#2}}}
\newcommand{\recsqrt}[1]{\frac{1}{\sqrt{#1}}}
\newcommand{\rec}[1]{\frac{1}{#1}}
\newcommand{\half}[1]{\frac{#1}{2}}
\newcommand{\res}[1]{\text{res}\lr{#1}}
\newcommand{\Epole}{E_{\text{pole}}}
\newcommand{\eq}[1]{(\ref{#1})}

\newcommand{\aC}[1]{a^{\dagger}_{#1}}
\newcommand{\aA}[1]{a_{#1}}
\newcommand{\cC}[1]{c^{\dagger}_{#1}}
\newcommand{\cA}[1]{c_{#1}}
\newcommand{\fC}[1]{f^{\dagger}_{#1}}
\newcommand{\fA}[1]{f_{#1}}
\newcommand{\bC}[1]{b^{\dagger}_{#1}}
\newcommand{\bA}[1]{b_{#1}}
\newcommand{\AugerC}[3]{\aC{#3,-#2}\fC{#1#2}\cA{#1,-#2}\cA{#1#2}}
\newcommand{\AugerA}[3]{\cC{#1#2}\cC{#1,-#2}\fA{#1#2}\aA{#3,-#2}}
\newcommand{\augerC}[2]{\fC{#1#2}\cA{#1#2}\cA{#1,-#2}}

\newcommand{\epsk}{\epsilon\lr{\v{k}}}
\newcommand{\epskA}{\epsilon\lr{\v{k}_A}}
\newcommand{\epskp}{\epsilon\lr{\v{k}_p}}
\newcommand{\epstilde}[1]{\tilde{\epsilon}\lr{#1}}
\newcommand{\epstildek}{\tilde{\epsilon}\lr{\v{k}}}
\newcommand{\epstildekp}{\tilde{\epsilon}\lr{\v{k}_1}}
\newcommand{\epstildekpp}{\tilde{\epsilon}\lr{\v{k}_2}}
\newcommand{\epsedge}{E_g}
\newcommand{\kmax}{k_{\text{max}}}
\newcommand{\sqrtEW}{\sqrt{\frac{E}{W}}}
\newcommand{\sqrtWE}{\sqrt{\frac{W}{E}}}
\newcommand{\Nel}{N_{\text{el}}}
\newcommand{\NVel}{N_V}
\newcommand{\epssumtwo}{s\lr{\v{k}_1,\v{k}_2}}
\newcommand{\epssumthree}{s\lr{\v{k},\v{k}_1,\v{k}_2}}

\newcommand{\vac}{\ket{0}}
\newcommand{\CSvac}{\ket{\v{0}}\vac}
\newcommand{\ketk}{\ket{\v{k}\sigma}}
\newcommand{\brak}{\bra{\v{k}\sigma}}
\newcommand{\ketEg}{\ket{E_0^{\lr{\NVel}}}}
\newcommand{\ketz}{\ket{z,\v{k}i\sigma}}
\newcommand{\braz}{\bra{z,\v{k}i\sigma}}

\newcommand{\commut}[2]{\left[#1,#2\right]}
\newcommand{\anticommut}[2]{\left\{#1,#2\right\}}
\newcommand{\norm}[1]{\left|#1\right|^2}
\newcommand{\abs}[1]{\left|#1\right|}
\newcommand{\vecnorm}[1]{\left\Vert#1\right\Vert}

\newcommand{\vx}{\lr{x}}
\newcommand{\vy}{\lr{y}}
\newcommand{\vk}{\lr{\v{k}}}
\newcommand{\vp}{\lr{\v{p}}}
\newcommand{\vq}{\lr{\v{q}}}
\newcommand{\vr}{\lr{\v{r}}}
\newcommand{\vt}{\lr{t}}
\newcommand{\vrt}{\lr{\v{r},t}}
\newcommand{\veps}{\lr{\epsilon}}
\newcommand{\vepsprime}{\lr{\epsilon'}}
\newcommand{\vtprime}{\lr{t'}}
\newcommand{\vE}{\lr{E}}
\newcommand{\vEt}{\lr{E,t}}
\newcommand{\vz}{\lr{z}}
\newcommand{\vn}{\lr{0}}
\newcommand{\vkA}{\lr{\v{k}_A}}
\newcommand{\vkp}{\lr{\v{k}_p}}
\newcommand{\vkAt}{\lr{\v{k}_A,t}}
\newcommand{\vkAkp}{\lr{\v{k}_A,\v{k}_p}}
\newcommand{\vkAkpt}{\lr{\v{k}_A,\v{k}_p,t}}
\newcommand{\vepskAkpt}{\lr{\epskA,\epskp,t}}
\newcommand{\vkAkpz}{\lr{\v{k}_A,\v{k}_p,z}}

\newcommand{\dipt}{d\lr{t}}
\newcommand{\dipCt}{d^*\lr{t}}
\newcommand{\dipki}{d_{\v{k}i}}
\newcommand{\dipCki}{d^*_{\v{k}i}}
\newcommand{\dipkit}{d_{\v{k}i}\lr{t}}
\newcommand{\dipkt}{d_{\v{k}}\lr{t}}
\newcommand{\dipCkt}{d^*_{\v{k}}\lr{t}}
\newcommand{\dipCkit}{d^*_{\v{k}i}\lr{t}}
\newcommand{\dipk}{d_0\vk}
\newcommand{\dipCk}{d^*_0\vk}

\newcommand{\cosp}{\cos\theta^+_{\v{k}}}
\newcommand{\cosm}{\cos\theta^-_{\v{k}}}
\newcommand{\sinp}{\sin\theta^+_{\v{k}}}
\newcommand{\sinm}{\sin\theta^-_{\v{k}}}
\newcommand{\cospm}{\cos\theta^{\pm}_{\v{k}}}
\newcommand{\sinpm}{\sin\theta^{\pm}_{\v{k}}}
\newcommand{\cosz}{\cos\theta^z_{\v{k}}}
\newcommand{\sinz}{\sin\theta^z_{\v{k}}}
\newcommand{\Ep}{E^+_{\v{k}}}
\newcommand{\Em}{E^-_{\v{k}}}
\newcommand{\Ez}{E^z_{\v{k}}}
\newcommand{\Ezprime}{E^{z'}_{\v{k}}}
\newcommand{\Epm}{E^{\pm}\vk}
\newcommand{\omegaz}{\omega_z\vk}
\newcommand{\omegapm}{\omega_{\pm}\vk}

\newcommand{\PsiNt}[2]{\Psi^{\lr{#1}}\lr{#2}}
\newcommand{\PsiMarked}[2]{\Psi^{\lr{#1}}_{#2}}
\newcommand{\PsiNtzn}[3]{\Psi^{\lr{#1}}_{#3}\lr{#2}}

\newcommand{\tint}{\int_0^t{dt'}~}
\newcommand{\kint}{\frac{a}{2\pi} \int_{-\pi/a}^{\pi/a} dk~}
\newcommand{\infint}{\int_0^\infty dt~}
\newcommand{\mpinfint}{\int_{-\infty}^{+\infty}}
\newcommand{\pminfint}{\int_{+\infty}^{-\infty}}
\newcommand{\Volint}{\int_V d^3r~}
\newcommand{\ksum}{\frac{1}{N}\sum_{\v{k}}}
\newcommand{\Nelsum}{\sum_{i=1}^{\Nel}}

\newcommand{\epskin}{\epsilon_{\text{kin}}}
\newcommand{\arctanh}{{\rm arctanh}}

\newcommand{\alphat}{\alpha\vt}
\newcommand{\alphadot}{\dot{\alpha}\vt}
\newcommand{\alphaddot}{\ddot{\alpha}\vt}
\newcommand{\betakit}{\beta_{\v{k}i}\vt}
\newcommand{\betakt}{\beta_{\v{k}}\vt}
\newcommand{\betak}{\beta_{\v{k}}}
\newcommand{\betadot}{\dot{\beta_{\v{k}}}\vt}
\newcommand{\Lalphap}{\overline{\alpha}\lr{p}}
\newcommand{\Lalphaz}{\overline{\alpha}\lr{z}}
\newcommand{\LalphaE}{\overline{\alpha}\lr{E}}

\newcommand{\Psisys}[1]{\ket{\Psi_{#1}^{\text{sys}}}}
\newcommand{\Psienv}[1]{\ket{\Psi_{#1}^{\text{env}}}}
\newcommand{\braPsisys}[1]{\bra{\Psi_{#1}^{\text{sys}}}}
\newcommand{\braPsienv}[1]{\bra{\Psi_{#1}^{\text{env}}}}
\newcommand{\Psisyst}[1]{\ket{\Psi_{\text{sys}}\lr{#1}}}
\newcommand{\Psienvt}[1]{\ket{\Psi_{\text{env}}\lr{#1}}}

\newcommand{\Esys}[1]{E^{\text{sys}}_{#1}}
\newcommand{\Eenv}[1]{E^{\text{env}}_{#1}}

\newcommand{\PsisysGS}{\Psi^{\text{sys}}_0}
\newcommand{\vPsisys}[1]{\left[\Psi^{\text{sys}}_{#1}\right]}

\newcommand{\erf}[1]{\text{erf}\lr{#1}}
\newcommand{\erfc}[1]{\text{erfc}\lr{#1}}

\newcommand{\supenv}{^{\text{env}}}
\newcommand{\subenv}{_{\text{env}}}
\newcommand{\subath}{_{\text{bath}}}
\newcommand{\supsys}{^{\text{sys}}}
\newcommand{\subsys}{_{\text{sys}}}
\newcommand{\supzero}{^{\lr{0}}}
\newcommand{\subzero}{_{\lr{0}}}
\newcommand{\submin}{_{\text{min}}}
\newcommand{\submax}{_{\text{max}}}
\newcommand{\supmin}{^{\text{min}}}
\newcommand{\supmax}{^{\text{max}}}
\newcommand{\E}[2]{E^{\lr{#2}}_{#1}}

\newcommand{\green}[1]{\langle\langle#1\rangle\rangle}

\newcommand{\kAsA}{\v{k}_A\sigma_A}
\newcommand{\kpsp}{\v{k}_p\sigma_p}

%\usepackage{subfiles}

\usepackage[toc]{appendix}
\usepackage{geometry}
\geometry{a4paper,total={170mm,257mm},left=25mm,top=15mm,bottom=25mm,right=15mm}
\pagestyle{plain} % no header on each page

\title{VUMPS-Strukturfaktor mit Einheitszellen}

\begin{document}

\maketitle

\section{Allgemein}

\begin{itemize}
\item $m,n$ geht \"uber verschiedene Einheitszellen
\item $j_x,j'_x,j_y,j'_y$ gehen innerhalb der Zelle von $0$ bis $L_x-1$ bzw. $L_y-1$
%\item $L_c$ ist die Zellengr\"o{\ss}e
\end{itemize}

\begin{equation}
\begin{split}
s^{\alpha\beta}\lr{\v{k}} &:=\frac{1}{\abs{\mathbb{Z}}}  \frac{1}{L_xL_y} \sum_{m,n\in\mathbb{Z}} \sum_{j_x,j'_x} \sum_{j_y,j'_y} e^{-ik_x\lr{nL_x-mL_x+j_x-j'_x}} e^{-ik_y\lr{j_y-j'_y}} \avg{O^{\beta\dagger}_{n,j_x,j_y} O^{\alpha}_{m,j'_x,j'_y}}\\
                      &= \frac{1}{\abs{\mathbb{Z}}} \sum_{m,n\in\mathbb{Z}} \frac{1}{L_x} \sum_{j_x,j'_x} e^{-ik_x\lr{nL_x-mL_x+j_x-j'_x}} \avg{O^{\beta\dagger}_{n,j_x}\lr{k_y} O^{\alpha}_{m,j'_x}\lr{k_y}}\\
                      &= \sum_{n\in\mathbb{Z}} \frac{1}{L_x} \sum_{j_x,j'_x} e^{-ik_x\lr{nL_x+j_x-j'_x}} \avg{O^{\beta\dagger}_{n,j_x}\lr{k_y} O^{\alpha}_{0,j'_x}\lr{k_y}}\\
                      &= \frac{1}{L_x} \sum_{j_x,j'_x} e^{-ik_x\lr{j_x-j'_x}} \cdot \lr{ \sum_{n\in\mathbb{Z}} e^{-ik_xnL_x} \avg{O^{\beta\dagger}_{n,j_x}\lr{k_y} O^{\alpha}_{0,j'_x}\lr{k_y}} }\\
                      &=: \frac{1}{L_x} \sum_{j_x,j'_x} e^{-ik_x\lr{j_x-j'_x}} \cdot s^{\alpha\beta}_{j_xj'_x}\lr{\v{k}}
\end{split}
\end{equation}

Fourier-Transformation bzgl. $y$:

\begin{equation}
O^{\alpha,\beta}_{n,j'_x}\lr{k_y} := \frac{1}{\sqrt{L_y}} \sum_{j'_y} e^{+ik_yj'_y} O^{\alpha,\beta}_{n,j'_x,j'_y}
\end{equation}

Strukturfaktor zwischen Einheitszellen:

\begin{equation}
s^{\alpha\beta}_{j_xj'_x}\lr{\v{k}} := \sum_{n\in\mathbb{Z}} e^{-ik_xnL_x} \avg{O^{\beta\dagger}_{n,j_x}\lr{k_y} O^{\alpha}_{0,j'_x}\lr{k_y}}
\end{equation}

Zerf\"allt in $n=0$, $n>0$, $n<0$:

\begin{equation}
\begin{split}
&s^{\alpha\beta}_{j_xj'_x}\lr{\v{k}} =\\
                           &= \avg{O^{\beta\dagger}_{0,j_x}\lr{k_y} O^{\alpha}_{0,j'_x}\lr{k_y}} + 
                               \sum_{n\in\mathbb{N}} e^{-ik_xnL_x} \avg{O^{\beta\dagger}_{n,j_x}\lr{k_y} O^{\alpha}_{0,j'_x}\lr{k_y}} +
                               \sum_{n\in\mathbb{N}} e^{+ik_xnL_x} \avg{O^{\beta\dagger}_{-n,j_x}\lr{k_y} O^{\alpha}_{0,j'_x}\lr{k_y}} \\
                            &= \avg{O^{\beta\dagger}_{0,j_x}\lr{k_y} O^{\alpha}_{0,j'_x}\lr{k_y}} + 
                               \sum_{n\in\mathbb{N}} e^{-ik_xnL_x} \avg{O^{\alpha\dagger}_{0,j'_x}\lr{k_y} O^{\beta}_{n,j_x}\lr{k_y}}^* + 
                               \sum_{n\in\mathbb{N}} e^{+ik_xnL_x} \avg{O^{\beta\dagger}_{0,j_x}\lr{k_y} O^{\alpha}_{n,j'_x}\lr{k_y}}\\
                            &= s^{\alpha\beta}_{j_xj'_x,\text{cell}}\lr{k_y} + s^{\alpha\beta}_{j_xj'_x,\text{inter}}\lr{\v{k}}
\end{split}
\end{equation}

Es wird $s^{\alpha\beta}_{j_xj'_x,\text{inter}}\lr{\v{k}}$ von \texttt{intercellSF} (mehrere $q$-Werte), \texttt{intercellSFpoint} (ein $q$-Wert) berechnet. Die volle Transformation geschieht mit \texttt{SF} und \texttt{SFpoint}.

Au{\ss}erdem scheint die Annahme $\lr{O^{\alpha,\beta}}^{\dagger}=O^{\alpha,\beta}$, bzw. $O^{\alpha\dagger}O^{\beta} = O^{\beta\dagger}O^{\alpha}$ getroffen zu werden, da nur das Produkt vorkommt und man die konjugierten Operatoren nicht \"ubergibt. In ``Tangent-space methods for uniform matrix product states'' (Kap. 2.5) verschwinden dann einfach die Kreuze... F\"ur Operatoren, die das nicht erf\"ullen, m\"usste man das ganze nochmal pr\"ufen.

F\"ur $j_x=j'_x$ muss eigentlich nur ein Term berechnet werden:

\begin{equation}
\begin{split}
s_{j_xj_x,\text{inter}}\lr{\v{k}} &= \sum_{n\in\mathbb{N}} \lr{e^{-ik_xnL_x}\avg{O^{\dagger}_{0,j_x}\lr{k_y} O_{n,j_x}\lr{k_y}}^* + e^{+ik_xnL_x}\avg{O^{\dagger}_{0,j_x}\lr{k_y} O_{n,j_x}\lr{k_y}}}\\
                                                &= 2\text{Re} \sum_{n\in\mathbb{N}} e^{+ik_xnL_x} \avg{O_{0,j_x}^{\dagger}\lr{k_y} O_{n,j_x}\lr{k_y}}
\end{split}
\end{equation}

Die Gleichheit der Terme sieht man numerisch, wird aber nicht ausgenutzt.

\section{Fourier-Transformation in $y$-Richtung}

Die Fourier-Transformation in $y$-Richtung geschieht mit \texttt{Geometry2D::FTy\_phases}. Die Argumente sind $j_x$, der Index von $k_y$ (\"uber $k_y=2\pi/L_y\cdot n$) und das Vorzeichen der Phase (nach der oberen Konvention positiv f\"ur $O$ und negativ f\"ur $O^{\dagger}$).
Die Funktion liefert $L_x \cdot L_y$ Koeffizienten der Linearkombination, die man dann mit \texttt{Mpo::setLocalSum} blind aufsummieren kann. Der Normierungsfaktor ist mitverarbeitet. Zum Beispiel mit $L_x=L_y=2$, $k_y=\pi$, $j_x=1$ und positiver Phase:

\begin{equation}
\frac{1}{\sqrt{L_y}} \sum_{j_y} e^{+i\pi j_y} O_{1,j_y} = \frac{1}{\sqrt{L_y}} \lr{ O_{1,0} - O_{1,1} }
\end{equation}

In der \texttt{SNAKE}-Geometrie wird $\lr{x,y}=\lr{1,0}$ auf $i=3$ gemappt und $\lr{x,y}=\lr{1,1}$ auf $i=2$ (bei der \texttt{CHESSBOARD}-Geometrie umgekehrt). Das Ergebnis von \texttt{FTy\_phases} ist also:

\begin{equation}
\begin{split}
i&=0: 0\\
i&=1: 0\\
i&=2: -\frac{1}{\sqrt{2}}\\
i&=3: +\frac{1}{\sqrt{2}}
\end{split}
\end{equation}


%F\"ur einen AFM sieht das ganze jetzt so aus:

%\begin{center}
%\includegraphics[scale=0.4]{AFM}
%\end{center}

%F\"ur einen FM ist das Maximum allerdings nicht ganz bei $q=0$... Liegt es vielleicht daran, dass der Zustand zu schlecht ist (Ferromagnetismus im $S=0$-Unterraum)?...

%\begin{center}
%\includegraphics[scale=0.4]{FM}
%\end{center}

\end{document}
