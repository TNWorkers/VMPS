\documentclass[12pt,paper=a4]{article}
\usepackage{ifxetex}
\ifxetex
	\usepackage{fontspec}
	\usepackage{libertine}
\else
	\usepackage[T1]{fontenc}
	\usepackage[utf8]{inputenc}
	\usepackage{lmodern}
\fi

\usepackage{datetime}
\usepackage{amsmath}
\usepackage{amsthm}
\usepackage{amssymb}
\usepackage{subfigure}
\usepackage{graphicx}
\usepackage{caption}
\usepackage[dvips,letterpaper,left=1.0in,top=0.35in,right=1.0in,bottom=0.75in]{geometry}
\usepackage{dsfont}
\usepackage{tikz}
\usepackage{algorithm}
%\usepackage{url}
\usepackage[style=authoryear,
            maxbibnames=99, %list all authors in fullcite
            %citestyle=authoryearbrackets,
            uniquename=false, %no first names
            uniquelist=false,  %no first names
            %sorting=nymt, %sorting according to name, year, month, t? (custom scheme in authoryearbrackets.cbx)
            firstinits=true,
            url=false,doi=false,isbn=false
           ]{biblatex}
\AtEveryCite{\renewcommand{\mkbibnamelast}[1]{\mkbibemph{#1}‌​}} %last names in italics
\renewbibmacro{in:}{} %suppresses "in:" before journal
\bibliography{/afs/physnet.uni-hamburg.de/users/th1_po/rrausch/PHDTEX/main/Rausch_2016.bib}

% enumerate bibliography items despite 'authoryear' citation style
\defbibenvironment{bibliography}
  {\enumerate
     {}
     {\setlength{\leftmargin}{\bibhang}%
      \setlength{\itemindent}{-\leftmargin}%
      \setlength{\itemsep}{\bibitemsep}%
      \setlength{\parsep}{\bibparsep}}}
  {\endenumerate}
  {\item}

%\bibliographystyle{eng-own}
\usepackage[noend]{algpseudocode}
\usetikzlibrary{decorations.markings}
\usetikzlibrary{decorations.pathmorphing}
\usetikzlibrary{decorations.pathreplacing}

%\usepackage{hyperref}
%\hypersetup{colorlinks=true,bookmarksopen=true,bookmarksopenlevel=3,allcolors=blue}
%\usepackage{hypcap}
\usepackage{pdfpages}

\setlength{\parindent}{0ex}
\setlength{\parskip}{1.3ex plus 0.2ex minus 0.2ex}

%\newcommand{\dirIntro}{/afs/physnet.uni-hamburg.de/users/th1_po/rrausch/PHDTEX/Intro/}
%\newcommand{\dirED}{/afs/physnet.uni-hamburg.de/users/th1_po/rrausch/PHDTEX/ED/}
%\newcommand{\dirLanczos}{/afs/physnet.uni-hamburg.de/users/th1_po/rrausch/PHDTEX/Lanczos/}
%\newcommand{\dirDMRG}{/afs/physnet.uni-hamburg.de/users/th1_po/rrausch/PHDTEX/DMRG/}
%\newcommand{\dirBetheAnsatz}{/afs/physnet.uni-hamburg.de/users/th1_po/rrausch/PHDTEX/BetheAnsatz/}
%\newcommand{\dirBetheAES}{/afs/physnet.uni-hamburg.de/users/th1_po/rrausch/PHDTEX/BetheAES/}
%\newcommand{\dirAESeq}{/afs/physnet.uni-hamburg.de/users/th1_po/rrausch/PHDTEX/AESeq/}
%\newcommand{\dirDoublonDyn}{/afs/physnet.uni-hamburg.de/users/th1_po/rrausch/PHDTEX/DoublonDyn/}
%\newcommand{\dirTwoParticleSpec}{/afs/physnet.uni-hamburg.de/users/th1_po/rrausch/PHDTEX/TwoParticleSpec/}

% hyperlinks for citations with biblatex:

\DeclareCiteCommand{\cite}
  {\usebibmacro{prenote}}
  {\usebibmacro{citeindex}%
   \printtext[bibhyperref]{\usebibmacro{cite}}}
  {\multicitedelim}
  {\usebibmacro{postnote}}

\DeclareCiteCommand*{\cite}
  {\usebibmacro{prenote}}
  {\usebibmacro{citeindex}%
   \printtext[bibhyperref]{\usebibmacro{citeyear}}}
  {\multicitedelim}
  {\usebibmacro{postnote}}

\DeclareCiteCommand{\parencite}[\mkbibparens]
  {\usebibmacro{prenote}}
  {\usebibmacro{citeindex}%
    \printtext[bibhyperref]{\usebibmacro{cite}}}
  {\multicitedelim}
  {\usebibmacro{postnote}}

\DeclareCiteCommand*{\parencite}[\mkbibparens]
  {\usebibmacro{prenote}}
  {\usebibmacro{citeindex}%
    \printtext[bibhyperref]{\usebibmacro{citeyear}}}
  {\multicitedelim}
  {\usebibmacro{postnote}}

\DeclareCiteCommand{\footcite}[\mkbibfootnote]
  {\usebibmacro{prenote}}
  {\usebibmacro{citeindex}%
  \printtext[bibhyperref]{ \usebibmacro{cite}}}
  {\multicitedelim}
  {\usebibmacro{postnote}}

\DeclareCiteCommand{\footcitetext}[\mkbibfootnotetext]
  {\usebibmacro{prenote}}
  {\usebibmacro{citeindex}%
   \printtext[bibhyperref]{\usebibmacro{cite}}}
  {\multicitedelim}
  {\usebibmacro{postnote}}

\DeclareCiteCommand{\textcite}
  {\boolfalse{cbx:parens}}
  {\usebibmacro{citeindex}%
   \printtext[bibhyperref]{\usebibmacro{textcite}}}
  {\ifbool{cbx:parens}
     {\bibcloseparen\global\boolfalse{cbx:parens}}
     {}%
   \multicitedelim}
  {\usebibmacro{textcite:postnote}}

\definecolor{asparagus}{rgb}{0.53, 0.66, 0.42}
\definecolor{goldenrod}{rgb}{0.85, 0.64, 0.125}
\newcommand{\qcite}[1]{\colorbox{asparagus}{[#1]}} %quick citation
\newcommand{\note}[1]{\colorbox{goldenrod}{[#1]}} %quick note
\newcommand{\qnote}[1]{\colorbox{goldenrod}{[#1]}} %quick note

\newcommand{\bi}{\begin{itemize}}
\newcommand{\ei}{\end{itemize}}
\newcommand{\be}{\begin{equation}}
\newcommand{\ee}{\end{equation}}
\newcommand{\bs}{\begin{split}}
\newcommand{\es}{\end{split}}

\let\vaccent=\v % rename builtin command \v{} to \vaccent{}
\renewcommand{\v}[1]{\ensuremath{\mathbf{#1}}} % for vectors
\newcommand{\ket}[1]{\big| #1 \big>} % for Dirac bras
\newcommand{\bra}[1]{\big< #1 \big|} % for Dirac kets
\newcommand{\braket}[2]{\big<#1 \vphantom{#2} \big| #2 \vphantom{#1} \big>} % for Dirac brackets
\newcommand{\matrixel}[3]{\big< #1 \vphantom{#2#3} \big| #2 \big| #3 \vphantom{#1#2} \big>} % for Dirac matrix elements
\newcommand{\pd}[2]{\frac{\partial #1}{\partial #2}} % for partial derivatives
\renewcommand{\d}[2]{\frac{d #1}{d #2}} % for derivatives
\newcommand{\gv}[1]{\ensuremath{\mbox{\boldmath$ #1 $}}} % for vectors of Greek letters
\renewcommand{\div}[1]{\gv{\nabla} \cdot #1} % for divergence
\newcommand{\curl}[1]{\gv{\nabla} \times #1} % for curl
\renewcommand{\=}[1]{\stackrel{#1}{=}} % for putting numbers above =
\newcommand{\avg}[1]{\big< #1 \big>} % for average

\newcommand{\lr}[1]{\left(#1\right)}
\newcommand{\lrsq}[1]{[#1]}
\newcommand{\sqrtfrac}[2]{\sqrt{\frac{#1}{#2}}}
\newcommand{\recsqrt}[1]{\frac{1}{\sqrt{#1}}}
\newcommand{\rec}[1]{\frac{1}{#1}}
\newcommand{\half}[1]{\frac{#1}{2}}
\newcommand{\res}[1]{\text{res}\lr{#1}}
\newcommand{\Epole}{E_{\text{pole}}}
\newcommand{\eq}[1]{(\ref{#1})}

\newcommand{\aC}[1]{a^{\dagger}_{#1}}
\newcommand{\aA}[1]{a_{#1}}
\newcommand{\cC}[1]{c^{\dagger}_{#1}}
\newcommand{\cA}[1]{c_{#1}}
\newcommand{\fC}[1]{f^{\dagger}_{#1}}
\newcommand{\fA}[1]{f_{#1}}
\newcommand{\bC}[1]{b^{\dagger}_{#1}}
\newcommand{\bA}[1]{b_{#1}}
\newcommand{\AugerC}[3]{\aC{#3,-#2}\fC{#1#2}\cA{#1,-#2}\cA{#1#2}}
\newcommand{\AugerA}[3]{\cC{#1#2}\cC{#1,-#2}\fA{#1#2}\aA{#3,-#2}}
\newcommand{\augerC}[2]{\fC{#1#2}\cA{#1#2}\cA{#1,-#2}}

\newcommand{\epsk}{\epsilon\lr{\v{k}}}
\newcommand{\epskA}{\epsilon\lr{\v{k}_A}}
\newcommand{\epskp}{\epsilon\lr{\v{k}_p}}
\newcommand{\epstilde}[1]{\tilde{\epsilon}\lr{#1}}
\newcommand{\epstildek}{\tilde{\epsilon}\lr{\v{k}}}
\newcommand{\epstildekp}{\tilde{\epsilon}\lr{\v{k}_1}}
\newcommand{\epstildekpp}{\tilde{\epsilon}\lr{\v{k}_2}}
\newcommand{\epsedge}{E_g}
\newcommand{\kmax}{k_{\text{max}}}
\newcommand{\sqrtEW}{\sqrt{\frac{E}{W}}}
\newcommand{\sqrtWE}{\sqrt{\frac{W}{E}}}
\newcommand{\Nel}{N_{\text{el}}}
\newcommand{\NVel}{N_V}
\newcommand{\epssumtwo}{s\lr{\v{k}_1,\v{k}_2}}
\newcommand{\epssumthree}{s\lr{\v{k},\v{k}_1,\v{k}_2}}

\newcommand{\vac}{\ket{0}}
\newcommand{\CSvac}{\ket{\v{0}}\vac}
\newcommand{\ketk}{\ket{\v{k}\sigma}}
\newcommand{\brak}{\bra{\v{k}\sigma}}
\newcommand{\ketEg}{\ket{E_0^{\lr{\NVel}}}}
\newcommand{\ketz}{\ket{z,\v{k}i\sigma}}
\newcommand{\braz}{\bra{z,\v{k}i\sigma}}

\newcommand{\commut}[2]{\left[#1,#2\right]}
\newcommand{\anticommut}[2]{\left\{#1,#2\right\}}
\newcommand{\norm}[1]{\left|#1\right|^2}
\newcommand{\abs}[1]{\left|#1\right|}
\newcommand{\vecnorm}[1]{\left\Vert#1\right\Vert}

\newcommand{\vx}{\lr{x}}
\newcommand{\vy}{\lr{y}}
\newcommand{\vk}{\lr{\v{k}}}
\newcommand{\vp}{\lr{\v{p}}}
\newcommand{\vq}{\lr{\v{q}}}
\newcommand{\vr}{\lr{\v{r}}}
\newcommand{\vt}{\lr{t}}
\newcommand{\vrt}{\lr{\v{r},t}}
\newcommand{\veps}{\lr{\epsilon}}
\newcommand{\vepsprime}{\lr{\epsilon'}}
\newcommand{\vtprime}{\lr{t'}}
\newcommand{\vE}{\lr{E}}
\newcommand{\vEt}{\lr{E,t}}
\newcommand{\vz}{\lr{z}}
\newcommand{\vn}{\lr{0}}
\newcommand{\vkA}{\lr{\v{k}_A}}
\newcommand{\vkp}{\lr{\v{k}_p}}
\newcommand{\vkAt}{\lr{\v{k}_A,t}}
\newcommand{\vkAkp}{\lr{\v{k}_A,\v{k}_p}}
\newcommand{\vkAkpt}{\lr{\v{k}_A,\v{k}_p,t}}
\newcommand{\vepskAkpt}{\lr{\epskA,\epskp,t}}
\newcommand{\vkAkpz}{\lr{\v{k}_A,\v{k}_p,z}}

\newcommand{\dipt}{d\lr{t}}
\newcommand{\dipCt}{d^*\lr{t}}
\newcommand{\dipki}{d_{\v{k}i}}
\newcommand{\dipCki}{d^*_{\v{k}i}}
\newcommand{\dipkit}{d_{\v{k}i}\lr{t}}
\newcommand{\dipkt}{d_{\v{k}}\lr{t}}
\newcommand{\dipCkt}{d^*_{\v{k}}\lr{t}}
\newcommand{\dipCkit}{d^*_{\v{k}i}\lr{t}}
\newcommand{\dipk}{d_0\vk}
\newcommand{\dipCk}{d^*_0\vk}

\newcommand{\cosp}{\cos\theta^+_{\v{k}}}
\newcommand{\cosm}{\cos\theta^-_{\v{k}}}
\newcommand{\sinp}{\sin\theta^+_{\v{k}}}
\newcommand{\sinm}{\sin\theta^-_{\v{k}}}
\newcommand{\cospm}{\cos\theta^{\pm}_{\v{k}}}
\newcommand{\sinpm}{\sin\theta^{\pm}_{\v{k}}}
\newcommand{\cosz}{\cos\theta^z_{\v{k}}}
\newcommand{\sinz}{\sin\theta^z_{\v{k}}}
\newcommand{\Ep}{E^+_{\v{k}}}
\newcommand{\Em}{E^-_{\v{k}}}
\newcommand{\Ez}{E^z_{\v{k}}}
\newcommand{\Ezprime}{E^{z'}_{\v{k}}}
\newcommand{\Epm}{E^{\pm}\vk}
\newcommand{\omegaz}{\omega_z\vk}
\newcommand{\omegapm}{\omega_{\pm}\vk}

\newcommand{\PsiNt}[2]{\Psi^{\lr{#1}}\lr{#2}}
\newcommand{\PsiMarked}[2]{\Psi^{\lr{#1}}_{#2}}
\newcommand{\PsiNtzn}[3]{\Psi^{\lr{#1}}_{#3}\lr{#2}}

\newcommand{\tint}{\int_0^t{dt'}~}
\newcommand{\kint}{\frac{a}{2\pi} \int_{-\pi/a}^{\pi/a} dk~}
\newcommand{\infint}{\int_0^\infty dt~}
\newcommand{\mpinfint}{\int_{-\infty}^{+\infty}}
\newcommand{\pminfint}{\int_{+\infty}^{-\infty}}
\newcommand{\Volint}{\int_V d^3r~}
\newcommand{\ksum}{\frac{1}{N}\sum_{\v{k}}}
\newcommand{\Nelsum}{\sum_{i=1}^{\Nel}}

\newcommand{\epskin}{\epsilon_{\text{kin}}}
\newcommand{\arctanh}{{\rm arctanh}}

\newcommand{\alphat}{\alpha\vt}
\newcommand{\alphadot}{\dot{\alpha}\vt}
\newcommand{\alphaddot}{\ddot{\alpha}\vt}
\newcommand{\betakit}{\beta_{\v{k}i}\vt}
\newcommand{\betakt}{\beta_{\v{k}}\vt}
\newcommand{\betak}{\beta_{\v{k}}}
\newcommand{\betadot}{\dot{\beta_{\v{k}}}\vt}
\newcommand{\Lalphap}{\overline{\alpha}\lr{p}}
\newcommand{\Lalphaz}{\overline{\alpha}\lr{z}}
\newcommand{\LalphaE}{\overline{\alpha}\lr{E}}

\newcommand{\Psisys}[1]{\ket{\Psi_{#1}^{\text{sys}}}}
\newcommand{\Psienv}[1]{\ket{\Psi_{#1}^{\text{env}}}}
\newcommand{\braPsisys}[1]{\bra{\Psi_{#1}^{\text{sys}}}}
\newcommand{\braPsienv}[1]{\bra{\Psi_{#1}^{\text{env}}}}
\newcommand{\Psisyst}[1]{\ket{\Psi_{\text{sys}}\lr{#1}}}
\newcommand{\Psienvt}[1]{\ket{\Psi_{\text{env}}\lr{#1}}}

\newcommand{\Esys}[1]{E^{\text{sys}}_{#1}}
\newcommand{\Eenv}[1]{E^{\text{env}}_{#1}}

\newcommand{\PsisysGS}{\Psi^{\text{sys}}_0}
\newcommand{\vPsisys}[1]{\left[\Psi^{\text{sys}}_{#1}\right]}

\newcommand{\erf}[1]{\text{erf}\lr{#1}}
\newcommand{\erfc}[1]{\text{erfc}\lr{#1}}

\newcommand{\supenv}{^{\text{env}}}
\newcommand{\subenv}{_{\text{env}}}
\newcommand{\subath}{_{\text{bath}}}
\newcommand{\supsys}{^{\text{sys}}}
\newcommand{\subsys}{_{\text{sys}}}
\newcommand{\supzero}{^{\lr{0}}}
\newcommand{\subzero}{_{\lr{0}}}
\newcommand{\submin}{_{\text{min}}}
\newcommand{\submax}{_{\text{max}}}
\newcommand{\supmin}{^{\text{min}}}
\newcommand{\supmax}{^{\text{max}}}
\newcommand{\E}[2]{E^{\lr{#2}}_{#1}}

\newcommand{\green}[1]{\langle\langle#1\rangle\rangle}

\newcommand{\kAsA}{\v{k}_A\sigma_A}
\newcommand{\kpsp}{\v{k}_p\sigma_p}

%\usepackage{subfiles}

\usepackage[toc]{appendix}
\usepackage{geometry}
\geometry{a4paper}
\pagestyle{plain} % no header on each page

\renewcommand{\c}[1]{c_{#1}}
\newcommand{\cdag}[1]{c^{\dagger}_{#1}}

\newcommand{\cs}[1]{c_{#1\sigma}}
\newcommand{\cdags}[1]{c^{\dagger}_{#1\sigma}}

\newcommand{\cS}[1]{c_{#1\overline{\sigma}}}
\newcommand{\cdagS}[1]{c^{\dagger}_{#1\overline{\sigma}}}

\newcommand{\cUP}[1]{c_{#1\uparrow}}
\newcommand{\cdagUP}[1]{c^{\dagger}_{#1\uparrow}}
\newcommand{\cDN}[1]{c_{#1\downarrow}}
\newcommand{\cdagDN}[1]{c^{\dagger}_{#1\downarrow}}

\newcommand{\nUP}[1]{n_{#1\uparrow}}
\newcommand{\nDN}[1]{n_{#1\downarrow}}

\newcommand{\Udag}[0]{U^{\dagger}}


\title{Hubbard model in momentum space using SU(2)$\times$U(1) symmetry}
\author{RR}

\begin{document}

\maketitle

\section{Fourier transform of hopping term}

Hopping term:
\begin{equation}
H = \sum_{ij\sigma} T_{ij} \cdags{i}\cs{j} = \lr{\vec{\cdag{}}}^{T} \cdot \underline{T} \cdot \vec{c}
\end{equation}

Diagonalize:
\begin{equation}
\begin{split}
\underline{T} &= \underline{U} \cdot \underline{D} \cdot \underline{U}^{\dagger}\\ 
\underline{D} &= \underline{U}^{\dagger} \cdot \underline{T} \cdot \underline{U}
\end{split}
\end{equation}
$\underline{U}$ contains the eigenvectors as column vectors.

\begin{equation}
\begin{split}
\lr{\vec{\cdag{}}}^{T} \cdot \underline{T} \cdot \vec{c} 
&= \lr{\vec{\cdag{}}}^{T} \cdot \underline{U} \cdot \underline{D} \cdot \underline{\Udag} \cdot \vec{c}\\
&= \sum_{\v{k}} \lr{\lr{\vec{\cdag{}}}^T \underline{U}}_{{\v{k}}} D_{{\v{k}}{\v{k}}} \lr{\underline{\Udag}\vec{c}}_{{\v{k}}}\\
&= \sum_{{\v{k}}} \lr{\sum_i \cdags{i} U_{i{\v{k}}}}  D_{{\v{k}}{\v{k}}} \lr{\sum_j \lr{\Udag}_{{\v{k}}j} \c{j}}
\end{split}
\end{equation}
with $D_{\v{k}\v{k}}=\epsilon\lr{\v{k}}$.

Define:
\begin{equation}
\begin{split}
\cs{\v{k}} &= \sum_j \lr{\Udag}_{{\v{k}}j} \cs{j} = \sum_j U^{*}_{j\v{k}} \cs{j}\\
\cdags{\v{k}} &= \sum_i U_{i\v{k}} \cdags{i}
\end{split}
\end{equation}

Inverse:
\begin{equation}
\begin{split}
\cs{j} &= \sum_{\v{k}} U_{j\v{k}} \cs{\v{k}}\\
\cdags{i} &= \sum_{\v{k}} U^*_{i\v{k}} \cdags{\v{k}}
\end{split}
\end{equation}

\section{Fourier transform of Hubbard term}

Hubbard term:
\begin{equation}
\begin{split}
H &= U\sum_{i} \cdagUP{i}\cUP{i}\cdagDN{i}\cDN{i}\\ 
&= U\sum_{i} \sum_{\v{k}\v{l}\v{m}\v{n}} U^*_{i\v{k}} U_{i\v{l}} U^*_{i\v{m}} U_{i\v{n}} \cdagUP{\v{k}}\cUP{\v{l}}\cdagDN{\v{m}}\cDN{\v{n}} \\
&= \sum_{\v{k}\v{l}\v{m}\v{n}} V_{\v{klmn}} \cdagUP{\v{k}}\cUP{\v{l}}\cdagDN{\v{m}}\cDN{\v{n}}
\end{split}
\end{equation}
with $V_{\v{klmn}}= U \sum_{i} U^*_{i\v{k}} U_{i\v{l}} U^*_{i\v{m}} U_{i\v{n}}$, which is easy to obtain numerically (throw out all zeros at this point). We now classify all interaction terms.

\subsection{Single-site}

\subsubsection{Hubbard interaction}

$V_{\v{iiii}}$ appears exactly $L$ times. Leads to a Hubbard term in k-space:
\begin{equation}
V_{\v{iiii}} \sim d_{\v{i}},
\end{equation}
where $d_{\v{i}}$ is the double occupancy in k-space.

\subsection{Two-site}

\subsubsection{Correlated hopping}

Terms of type:
\begin{equation}
V_{\v{jiii}} + V_{\v{ijii}} + V_{\v{iiji}} + V_{\v{iiij}} \sim \sum_{\sigma} \cdags{\v{j}} n_{\v{i}\overline{\sigma}} \cs{\v{i}} + h.c.
\end{equation}
Note: Because of $\lr{\c{\alpha}}^2=\lr{\cdag{\alpha}}^2=0$ we can add the other spin component to the n-operator: $n_{\v{i}\overline{\sigma}} \to n_{\v{i}}$, leading to:
\begin{equation}
V_{\v{jiii}} + V_{\v{ijii}} + V_{\v{iiji}} + V_{\v{iiij}} \sim \sum_{\sigma} \cdags{\v{j}} n_{\v{i}} \cs{\v{i}} + h.c.
\end{equation}

Result:
\begin{equation}
\boxed{ V_{\v{jiii}} + V_{\v{ijii}} + V_{\v{iiji}} + V_{\v{iiij}} \sim \mathtt{cdag\_nc}\lr{j,i} + \mathtt{cdagn\_c}\lr{i,j} }
\end{equation}

\subsubsection{Pair hopping}

Terms of type (no minus sign picked up):
\begin{equation}
V_{\v{ijij}} + V_{\v{jiji}} \sim \cdagUP{\v{i}} \cdagDN{\v{i}} \cdot \cDN{\v{j}} \cUP{\v{j}} + h.c.
\end{equation}

Result:
\begin{equation}
\boxed{ V_{\v{ijij}} + V_{\v{jiji}} \sim \mathtt{cdagcdag}\lr{i}\mathtt{cc}\lr{j} + \mathtt{cdagcdag}\lr{j}\mathtt{cc}\lr{i}
}
\end{equation}

\subsubsection{Spin exchange}

The spinflip terms are $V_{\v{ijji}}$ and $V_{\v{jiij}}$ (minus sign picked up!):
\begin{equation}
V_{\v{ijji}} \sim \cdagUP{\v{i}} \cUP{\v{j}} \cdagDN{\v{j}} \cDN{\v{i}} = -\cdagUP{\v{i}}\cDN{\v{i}} \cdagDN{\v{j}}\cUP{\v{j}} = -S^+_{\v{i}} S^-_{\v{j}}
\end{equation}

Must find the corresponding Ising terms. Consider:
\begin{equation}
\begin{split}
S^z_{\v{i}} S^z_{\v{j}} 
&= \frac{1}{2} \lr{\nUP{\v{i}}-\nDN{\v{i}}} \frac{1}{2}\lr{\nUP{\v{j}}-\nDN{\v{j}}}
= \frac{1}{4} \lr{ \nUP{\v{i}}\nUP{\v{j}} + \nDN{\v{i}}\nDN{\v{j}} -\nDN{\v{i}}\nUP{\v{j}} -\nUP{\v{i}}\nDN{\v{j}} }\\
\frac{1}{4} n_{\v{i}}n_{\v{j}} &= \frac{1}{4} \lr{ \nUP{\v{i}}\nUP{\v{j}} + \nDN{\v{i}}\nDN{\v{j}} +\nDN{\v{i}}\nUP{\v{j}} +\nUP{\v{i}}\nDN{\v{j}} }
\end{split}
\end{equation}

Subtract to remove same-spin terms:
\begin{equation}
S^z_{\v{i}} S^z_{\v{j}} - \frac{1}{4} n_{\v{i}}n_{\v{j}}
= -\frac{1}{2} \lr{ \nUP{\v{i}}\nDN{\v{j}} + \nDN{\v{i}}\nUP{\v{j}} }
\sim -\frac{1}{2} \lr{V_{\v{iijj}} + V_{\v{jjii}}}
\end{equation}

Sum everything up:
\begin{equation}
\boxed{ V_{\v{ijji}} + V_{\v{jiij}} + V_{\v{iijj}} + V_{\v{jjii}} 
\sim -2\lr{ \v{S}_{\v{i}}\cdot \v{S}_{\v{j}} - \frac{1}{4} n_{\v{i}}n_{\v{j}} }
}
\end{equation}

%%%%%%%%%%%%
\subsection{Three-site}
%%%%%%%%%%%%

\subsubsection{3-site correlated hopping}

Terms of type (no minus sign picked up):
\begin{equation}
V_{\v{iijk}} + V_{\v{iikj}} + V_{\v{jkii}} + V_{\v{kjii}} \sim \sum_{\sigma} n_{\v{i}\overline{\sigma}} \cdags{\v{j}}\cs{\v{k}} + h.c.
\end{equation}

Problem: Now we cannot just replace $n_{\v{i}\overline{\sigma}}$ by $n_{\v{i}}$ because a new term is generated. However, it turns out that this term is exactly canceled by the kinetic spin exchange term.

\begin{equation}
\boxed{ V_{\v{iijk}} + V_{\v{iikj}} + V_{\v{jkii}} + V_{\v{kjii}} + {\color{red}\text{same-spin terms}} \sim \mathtt{n}\lr{i} \mathtt{cdagc}\lr{j,k} + \mathtt{n}\lr{i} \mathtt{cdagc}\lr{k,j} 
}
\end{equation}

\subsubsection{Kinetic spin exchange}

Consider
\begin{equation}
\begin{split}
-\sum_{\sigma\overline{\sigma}} \cdags{\v{i}} \cS{\v{i}} \cdagS{\v{j}} \cs{\v{k}} 
&= \sum_{\sigma\overline{\sigma}} \cdags{\v{i}} \cs{\v{k}} \cdagS{\v{j}} \cS{\v{i}} \\
&\sim V_{\v{ikji}} + V_{\v{jiik}} - {\color{red}\text{same-spin terms}}
\end{split}
\end{equation}
The minus sign of the same-spin terms can be seen from the first expression for $\sigma=\overline{\sigma}$.

Now expand:
\begin{equation}
\begin{split}
-\sum_{\sigma\overline{\sigma}} \cdags{\v{i}} \cS{\v{i}} \cdagS{\v{j}} \cs{\v{k}} 
&= 
-\cdagUP{\v{i}} \cDN{\v{i}} \cdagDN{\v{j}} \cUP{\v{k}}
-\cdagDN{\v{i}} \cUP{\v{i}} \cdagUP{\v{j}} \cDN{\v{k}}
-\cdagUP{\v{i}} \cUP{\v{i}} \cdagUP{\v{j}} \cUP{\v{k}}
-\cdagDN{\v{i}} \cDN{\v{i}} \cdagDN{\v{j}} \cDN{\v{k}} \\
&= 
-S^+_{\v{i}} \cdagDN{\v{j}} \cUP{\v{k}} 
-S^-_{\v{i}} \cdagUP{\v{j}} \cDN{\v{k}}
-\nUP{\v{i}} \cdagUP{\v{j}} \cUP{\v{k}}
-\nDN{\v{i}} \cdagDN{\v{j}} \cDN{\v{k}}
\end{split}
\end{equation}
This is a nonlocal spin exchange term with a correlated-hopping offset (see: [\url{https://doi.org/10.1063/1.4944921}], Tab. II).

Result:
\begin{equation}
\boxed{
\begin{split} 
&V_{\v{ijki}} + V_{\v{kiij}} + V_{\v{jiik}} + V_{\v{ikji}} - {\color{red}\text{same-spin terms}} 
\sim \\
&\sqrt{3}\sqrt{3}\sqrt{2} \left[ \mathtt{S}\lr{i} \mathtt{cdagc3}\lr{j,k} + \mathtt{cdagc3}\lr{k,j} \mathtt{Sdag}\lr{i} \right]
-\frac{1}{2} \left[ \mathtt{n}\lr{i} \mathtt{cdagc}\lr{j,k} +  \mathtt{cdagc}\lr{k,j} \mathtt{n}\lr{i} \right]
\end{split}
}
\end{equation}

Note 1: One $\sqrt{3}$ factor cancels the $1/\sqrt{3}$ convention in $\mathtt{cdagc3}$.

Note 2: The offset term can be left out and the 3-site correlated-hopping term can be halved in compensation.

%%%%%%%%%%%%%%%%%%
\subsubsection{Pair creation/decay}
%%%%%%%%%%%%%%%%%%

Consider:
\begin{equation}
V_{\v{ijik}} = \cdagUP{\v{i}} \cUP{\v{j}} \cdagDN{\v{i}} \cDN{\v{k}} 
= - \cdagUP{\v{i}}\cdagDN{\v{i}} \cUP{\v{j}}\cDN{\v{k}}
\end{equation}

and: 
\begin{equation}
V_{\v{ikij}} = \cdagUP{\v{i}} \cUP{\v{k}} \cdagDN{\v{i}} \cDN{\v{j}} 
= +\cdagUP{\v{i}}\cdagDN{\v{i}} \cDN{\v{j}}\cUP{\v{k}}
\end{equation}

Summing up:

\begin{equation}
V_{\v{ijik}} + V_{\v{ikij}} 
= - \cdagUP{\v{i}}\cdagDN{\v{i}} \lr{ \cUP{\v{j}}\cDN{\v{k}} - \cDN{\v{j}}\cUP{\v{k}} } 
\end{equation}
The latter term are two annihilators coupled to a singlet.

Result:
\begin{equation}
\boxed{ 
V_{\v{ijik}} + V_{\v{ikij}} + V_{\v{jiki}} + V_{\v{kiji}} \sim \mathtt{cdagcdag}\lr{i} \mathtt{cc1}\lr{j,k} - \mathtt{cdagcdag1}\lr{j,k} \mathtt{cc}\lr{i}
}
\end{equation}
Note the minus sign and the same order $(j,k)$ of the indices.

%%%%%%%%%%%%%%
\subsection{Four-site terms}
%%%%%%%%%%%%%%

Consider:
\begin{equation}
\begin{split}
\lr{ \cdagUP{\v{i}}\cdagDN{\v{j}} - \cdagDN{\v{i}}\cdagUP{\v{j}} } 
\lr{ \cdag{\v{k}}\cDN{\v{l}} - \cDN{\v{k}}\cUP{\v{l}} }
&=
 \cdagUP{\v{i}} \cdagDN{\v{j}} \cUP{\v{k}} \cDN{\v{l}}
+\cdagDN{\v{i}} \cdagDN{\v{j}} \cDN{\v{k}} \cDN{\v{l}}
-\cdagDN{\v{i}} \cdagUP{\v{j}} \cUP{\v{k}} \cDN{\v{l}}
-\cdagUP{\v{i}} \cdagDN{\v{j}} \cDN{\v{k}} \cUP{\v{l}}\\
&= 
-\cdagUP{\v{i}} \cUP{\v{k}} \cdagDN{\v{j}} \cDN{\v{l}}
-\cdagDN{\v{i}} \cDN{\v{l}} \cdagDN{\v{j}} \cDN{\v{k}} 
-\cdagDN{\v{i}} \cUP{\v{k}} \cdagUP{\v{j}} \cDN{\v{l}}
-\cdagUP{\v{i}} \cUP{\v{l}} \cdagDN{\v{j}} \cDN{\v{k}}\\
\sim -\lr{ V_{\v{ikjl}} + V_{\v{jlik}} + V_{\v{jkil}} + V_{\v{iljk}} }
\end{split}
\end{equation}

Analogously for the h.c. term.

Result:
\begin{equation}
\boxed{
\begin{split} 
 &V_{\v{ikjl}} + V_{\v{jlik}} + V_{\v{jkil}} + V_{\v{iljk}}+V_{\v{kilj}} + V_{\v{ljki}} + V_{\v{kjli}} + V_{\v{likj}}
\sim \\
&\mathtt{cdagcdag}\lr{i,j} \mathtt{cc}\lr{k,l} + \mathtt{cdagcdag}\lr{l,k} \mathtt{cc}\lr{j,i}
\end{split}
}
\end{equation}

This means that the number of 4-site terms should be divisible by 8.

\end{document}
